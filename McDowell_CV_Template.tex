%% The MIT License (MIT)
%%
%% Copyright (c) 2015 Daniil Belyakov
%%
%% Permission is hereby granted, free of charge, to any person obtaining a copy
%% of this software and associated documentation files (the "Software"), to deal
%% in the Software without restriction, including without limitation the rights
%% to use, copy, modify, merge, publish, distribute, sublicense, and/or sell
%% copies of the Software, and to permit persons to whom the Software is
%% furnished to do so, subject to the following conditions:
%%
%% The above copyright notice and this permission notice shall be included in all
%% copies or substantial portions of the Software.
%%
%% THE SOFTWARE IS PROVIDED "AS IS", WITHOUT WARRANTY OF ANY KIND, EXPRESS OR
%% IMPLIED, INCLUDING BUT NOT LIMITED TO THE WARRANTIES OF MERCHANTABILITY,
%% FITNESS FOR A PARTICULAR PURPOSE AND NONINFRINGEMENT. IN NO EVENT SHALL THE
%% AUTHORS OR COPYRIGHT HOLDERS BE LIABLE FOR ANY CLAIM, DAMAGES OR OTHER
%% LIABILITY, WHETHER IN AN ACTION OF CONTRACT, TORT OR OTHERWISE, ARISING FROM,
%% OUT OF OR IN CONNECTION WITH THE SOFTWARE OR THE USE OR OTHER DEALINGS IN THE
%% SOFTWARE.

% The font could be set to Windows-specific Calibri by using the 'calibri' option
\documentclass[]{mcdowellcv}

% For mathematical symbols
\usepackage{amsmath}

% Set applicant's personal data for header
\name{Felipe Andres Apablaza Cheuquepan}
\address{Jose Domingo Canas 705, Depto 125 \linebreak Santiago, Chile}
\contacts{+56 972555408 \linebreak ing.apablaza@gmail.com}

\begin{document}

	% Print the header
	\makeheader
	
	% Print the content
	\begin{cvsection}{Experiencia Laboral}
		\begin{cvsubsection}{Technical Lead}{Banco BICE}{Agosto 2020 -- Enero 2021}
			\begin{itemize}
				\item Liderar y coordinar al equipo de desarrollo, y coordinar los desarrollos necesarios de la célula con el resto  de las áreas de TI. 
				\item Realización prueba de concepto con el proveedor “TOC” para obtener el conocimiento y habilitar
				al equipo para implementar autenticación biométrica facial para el flujo no – cliente del
				onboarding de inversiones. 
			\end{itemize}
		\end{cvsubsection}
		
		\begin{cvsubsection}{Technical Lead}{Banco BCI}{Febrero 2019 -- Septiembre 2020}	
			Célula Conversión y Beneficios
			\begin{itemize}
				\item Coordinación diaria de los releases de los desarrollos junto a los TLs de otras células en las Apps de iOS y Android de BCI.
				\item Actualización del Certificado SSL en las Apps de iOS y Android de BCI para asegurar la continuidad operacional previo al vencimiento.
				\item Análisis y corrección de un bug productivo en Android, que impedia que al presionar sobre una notificacion push, ésta redireccione hacia el producto financiero relacionado.
			\end{itemize}
			Célula Planes
			\begin{itemize}
				\item Liderar y coordinar al equipo de desarrollo en la construcción, velando por un balance apropiado entre calidad  técnica y velocidad de entrega de las soluciones, gestionando la deuda técnica generada.
				\item Mejora en el cumplimiento del scope del Q3 2019, de un 70\% a un 110\%, provocada por mejoras en los refinamientos técnicos, entre otras.
				\item Levantamiento, presentación y entrega a la Gerencia de Mantención, la documentación técnica del Viaje de Planes (arquitectura multinivel, dependencias, bases de datos, proceso batch, etc).
				\item Presentación de la arquitectura de las iniciativas del Trimestre al Equipo Extendido (Arquitectura de Soluciones, Riesgo Operacional y Seguridad) para determinar factibilidad técnica.
			\end{itemize}
		\end{cvsubsection}
		
		\begin{cvsubsection}{Developer}{Falabella Financiero}{Junio 2017 -- Enero 2019}		
			Célula de Reclamos
			\begin{itemize}
				\item Coordinación con las áreas de Fraude y Soporte de fraude para realizar un levantamiento de requisitos técnicos, con el fin de integrar el Viaje con el motor de reglas PayTrue.
				\item Integración Login de Active Directory para un Frontend enfocado a los ejecutivos del Call Center, desarrollado en Angular 4, usando la librería Adal.
				\item Desarrollo de varios microservicios en NodeJS, con sus correspondientes pruebas unitarias y despliegues en pipelines CI/CD, en Docker, superando un 90\% de covertura. Uno de ellos fue un motor de reglas, para posibilitar la evaluación de reclamos en línea. Las reglas estaban representadas en JSON.
				\item Implementación de nuevas funcionalidades para las Apps iOS de CMR Chile, Banco Falabella Perú, CMR Argentina y Falabella Mexico, usando el patrón MVP. Pruebas unitarias superaban el 80\% de cobertura. Se usaron las tecnologías Objective-C, Swift, AFNetworking, Alamofire y XCTest.
			\end{itemize}
		\end{cvsubsection}
		
		\begin{cvsubsection}{Software Engineer}{Outlier SpA}{Abril 2016 -- Junio 2017}
			\begin{itemize}
				\item Desarrollo Engine de renderización en Python, PyOpenGL, Vertex Shaders y Fragment Shaders en GLSL y NumPy, para soportar un visualizador de objetos enfocado a la industria Minera. Este motor estaba abstraído del Backend de renderización, y era posible incorporar nuevos tipos de objetos usando Polimorfismo.
				\item Reimplementación del motor de renderización en C++, GLM, OpenGL 3.X y OpenGL ES 2 para mejorar performance, preservando los atributos arquitectónicos anteriores, incorporando la portabilidad de plataforma.
				\item Diseño e implementación algoritmo geométrico, que permitía hacer un corte de los objetos mediante un plano, definido por el usuario.
				\item Integración de este Engine con aplicaciones de escritorio en Python, usando Boost.Python para generar el glue-code necesario.
				\item Integración de este Engine con aplicaciones Web, mediante la compilación a ASM.JS / WebAssembly usando el compilador Emscripten.
				\item Integración del Engine con aplicaciones WPF en .NET, mediante la generación de un Assembly nativo en el lenguaje C++/CLI, usando la API Nativa de Windows (WinAPI) para apoyar en esta integración.
			\end{itemize}
		\end{cvsubsection}

		\begin{cvsubsection}{Ingeniero Desarrollo}{Adexus}{Agosto 2012 -- Abril 2016}		
			\begin{itemize}
				\item Courses: Advanced Java III, Software Engineering, Mathematical Foundations of Computer Science I \& II.
				\item Promoted to Head TA in Fall 2004; led weekly meetings and supervised four other TAs.
			\end{itemize}
		\end{cvsubsection}
		\begin{cvsubsection}{Encargado de Informática}{YMCA Temuco}{Julio 2010 -- Agosto 2012}		
			\begin{itemize}
				\item Courses: Advanced Java III, Software Engineering, Mathematical Foundations of Computer Science I \& II.
				\item Promoted to Head TA in Fall 2004; led weekly meetings and supervised four other TAs.
			\end{itemize}
		\end{cvsubsection}
	\end{cvsection}
	
	\begin{cvsection}{Education}
		\begin{cvsubsection}{Temuco}{Universidad Católica de Temuco}{2005 -- 2012}
			\begin{itemize}
				\item Egresado de Ingeniería Civil en Informática.
				\item 
			\end{itemize}
		\end{cvsubsection}
	\end{cvsection}
	
	\begin{cvsection}{Technical Experience}
		\begin{cvsubsection}{Projects}{}{}
			\begin{itemize}
				\item \textbf{Multi-User Drawing Tool} (2004). Electronic classroom where multiple users can view and simultaneously draw on a “chalkboard” with each person’s edits synchronized.  C++, MFC
				\item \textbf{Synchronized Calendar} (2003 – 2004). Desktop calendar with globally shared and synchronized calendars, allowing users to schedule meetings with other users.  C\#.NET, SQL, XML
				\item \textbf{Operating System} (2002).  UNIX-style OS with scheduler, file system, text editor and calculator. C
			\end{itemize}
		\end{cvsubsection}
	\end{cvsection}
	
	\begin{cvsection}{Additional Experience and Awards}
		\begin{cvsubsection}{}{}{}	
			\begin{itemize}
				\item \textbf{Instructor (2003 – 2005):} Taught two full-credit Computer Science courses; average ratings of 4.8 out of 5.0.
				\item \textbf{Third Prize, Senior Design Projects:} Awarded 3rd prize for Synchronized Calendar project, out of 100 projects.
			\end{itemize}
		\end{cvsubsection}
	\end{cvsection}
	
	\begin{cvsection}{Languages and Technologies}
		\begin{cvsubsection}{}{}{}	
			\begin{itemize}
				\item C++; C; Java; Objective-C; C\#.NET; SQL; JavaScript; XSLT; XML (XSD) Schema 
				\item Visual Studio; Microsoft SQL Server; Eclipse; XCode; Interface Builder
			\end{itemize}
		\end{cvsubsection}
	\end{cvsection}
	
\end{document}
