%% The MIT License (MIT)
%%
%% Copyright (c) 2015 Daniil Belyakov
%%
%% Permission is hereby granted, free of charge, to any person obtaining a copy
%% of this software and associated documentation files (the "Software"), to deal
%% in the Software without restriction, including without limitation the rights
%% to use, copy, modify, merge, publish, distribute, sublicense, and/or sell
%% copies of the Software, and to permit persons to whom the Software is
%% furnished to do so, subject to the following conditions:
%%
%% The above copyright notice and this permission notice shall be included in all
%% copies or substantial portions of the Software.
%%
%% THE SOFTWARE IS PROVIDED "AS IS", WITHOUT WARRANTY OF ANY KIND, EXPRESS OR
%% IMPLIED, INCLUDING BUT NOT LIMITED TO THE WARRANTIES OF MERCHANTABILITY,
%% FITNESS FOR A PARTICULAR PURPOSE AND NONINFRINGEMENT. IN NO EVENT SHALL THE
%% AUTHORS OR COPYRIGHT HOLDERS BE LIABLE FOR ANY CLAIM, DAMAGES OR OTHER
%% LIABILITY, WHETHER IN AN ACTION OF CONTRACT, TORT OR OTHERWISE, ARISING FROM,
%% OUT OF OR IN CONNECTION WITH THE SOFTWARE OR THE USE OR OTHER DEALINGS IN THE
%% SOFTWARE.

% The font could be set to Windows-specific Calibri by using the 'calibri' option
\documentclass[]{mcdowellcv}

% For mathematical symbols
\usepackage{amsmath}

% Set applicant's personal data for header
\name{Felipe Andres Apablaza Cheuquepan}
\address{Ñuñoa \linebreak Región Metropolitana de Santiago, Chile}
\contacts{+56 972555408 \linebreak ing.apablaza@gmail.com}

\begin{document}

	% Print the header
	\makeheader
	
	% Print the content
	\begin{cvsection}{Experiencia Laboral}
		\begin{cvsubsection}{Technical Lead}{Banco BICE}{Agosto 2020 -- Enero 2021}
			\begin{itemize}
				\item Punto de contacto entre las áreas de TI y la célula de desarrolllo.
				\item Realización prueba de concepto para integrar una autenticación biométrica facial en el flujo no – cliente del
				OnBoarding de BICE Inversiones. 
			\end{itemize}
		\end{cvsubsection}
		
		\begin{cvsubsection}{Technical Lead}{Banco BCI}{Febrero 2019 -- Julio 2020}	
			Célula Conversión y Beneficios
			\begin{itemize}
				\item Coordinación diaria de los releases de los desarrollos junto a los TLs de otras células en las Apps de iOS y Android de BCI.
				\item Actualización del Certificado SSL en las Apps de iOS y Android de BCI para asegurar la continuidad operacional previo al vencimiento.
				\item Análisis y corrección de un bug productivo en Android, que impedia que al presionar sobre una notificacion push, ésta redireccione hacia el producto financiero relacionado.
			\end{itemize}
			Célula Planes
			\begin{itemize}
				\item Liderar y coordinar al equipo de desarrollo en la construcción, velando por un balance apropiado entre calidad técnica y velocidad de entrega de las soluciones, gestionando la deuda técnica generada.
				\item Evangelización de buenas prácticas en la célula, tales como trazabilidad, refinamientos técnicos, y diseño técnico de HdU.
				\item Levantamiento de la documentación técnica del Viaje de Planes (arquitectura multinivel, dependencias, bases de datos, procesos batch, etc). Presentación y entrega a la Gerencia de Mantención.
				\item Presentación de la arquitectura asociada de las iniciativas del Trimestre al Equipo Extendido (Arquitectura de Soluciones, Riesgo Operacional y Seguridad) para determinar factibilidad técnica.
			\end{itemize}
		\end{cvsubsection}
		
		\begin{cvsubsection}{Developer}{Falabella Financiero}{Junio 2017 -- Enero 2019}		
			Célula de Reclamos
			\begin{itemize}
				\item Participación en desarrollos varios cono parte de la Célula de Reclamos, trabajando con Scrum, y usando herramientas Atlassian (Jira y Confluence).
				\item Levantamiento de requisitos técnicos y funcionales para integrar el motor de reglas corporativo en el viaje de Reclamos.
				\item Integración Login de Active Directory en un Frontend enfocado para los ejecutivos de Call Center. Angular 4, Adal.
				\item Desarrollo de microservicios en NodeJS, siendo uno de ellos un mini motor de reglas de fraude para movimientos de tarjetas. JavaScript con pruebas unitarias y Docker.
				\item Mejoras en las Apps iOS de CMR Chile, Banco Falabella Perú, CMR Argentina y Falabella Mexico. Se usó el patrón MVP con los lenguajes Objective-C y Swift. Pruebas unitarias con XCTest.
			\end{itemize}
		\end{cvsubsection}
		
		\begin{cvsubsection}{Software Engineer}{Outlier SpA}{Abril 2016 -- Junio 2017}
			\begin{itemize}
				\item Diseño de una arquitectura modular y extensible para tipos de modelos y backends de renderización de un motor de visualización 3D, encofado para el sector minero.
				\item Implementación e integración de motor de visualización en un software de escritorio. Python, PyOpenGL, GLSL y NumPy.
				\item Optimización performance del motor de visualización en C++, GLM, OpenGL 3.X y OpenGL ES 2, Boost.Python.
				\item Diseño e implementación algoritmo geométrico de corte de geometría, definido por el usuario.
				\item Cross compilación hacia Web en ASM.JS / WebAssembly, usando el compilador Emscripten.
				\item Integración del motor en WPF en .NET, generando un assembly con C++/CLI y Win32 API para hacer la interoperación.
			\end{itemize}
		\end{cvsubsection}

		\begin{cvsubsection}{Ingeniero Desarrollo}{Adexus}{Agosto 2012 -- Abril 2016}		
			\begin{itemize}-
				\item Mantenimiento Software de Facturación Electrónica. Java 6.
				\item Optimización fragmentación de memoria RAM durante la carga de boletas y facturas.
				\item Optimización performance en carga de receptores electrónicos del SII.
				\item Diseño e implementación motor de reglas para aceptar o rechazar facturas recibidas.
			\end{itemize}
		\end{cvsubsection}

		\begin{cvsubsection}{Encargado de Informática}{YMCA Temuco}{Julio 2010 -- Agosto 2012}		
			\begin{itemize}
				\item Mantenimiento software gestor de socios. Visual Basic 6.
				\item Diseño de arquitectura e implementación de un prototipo de sistema distribuido de control de acceso biométrico, sobre TCP/IP, permitiendo el ingreso automático de los socios que se encuentren al día en sus pagos.
				\item Diseño e implementación subsistema de gestión y reservas de canchas de pasto sintético.
				\item Diseño e implementación subsistema de evaluación antropométrica, en la que se entrega una estimación de la composición corporal del socio en base a sus medidas.
				\item Diseño arquitectura (componentes de software y modelo de base de datos) nuevo sistema gestor de socios.
			\end{itemize}
		\end{cvsubsection}
	\end{cvsection}
	
	\begin{cvsection}{Educación}
		\begin{cvsubsection}{Temuco}{Universidad Católica de Temuco}{2005 -- 2012, 2021}
			\begin{itemize}
				\item (2005 - 2012) Egresado de Ingeniería Civil en Informática.
				\item (2021) Alumno Tesista de Ingeniería Civil en Informática.
			\end{itemize}
		\end{cvsubsection}
	\end{cvsection}
	
	\begin{cvsection}{Experiencia Técnica}
		\begin{cvsubsection}{Desarrollos varios}{}{}
			\begin{itemize}
				\item \textbf{Cursos en Tablet Android} (2014). Desarrollo Aplicación Android 4 para mostrar cursos en video, incluyendo indexación en tablas de contenidos y evaluación con alternativas al final del curso. Java.
				\item \textbf{Estimador de Riesgo Corto Plazo} (2019). Desarrollo aplicación de escritorio para mostrar estimaciones de riesgos en minería en base a data preexistente. Csharp, WPF.
				\item \textbf{Mejoras Estimador de Riesgo Largo Plazo} (2018). Mantenimiento de una base de código preexistente, en la que se mejoran gráficos. C++11 y Qt 5.
				\item \textbf{Mejoras Simulador Flujo Gravitacional} (2020). Mantenimiento de una base de código preexistente, en la que se agregan nuevos atributos, parámetros de simulación visualización tridimensional y migración a MFC. C++11, Qt 5, OpenSceneGraph, WinAPI y MFC.
			\end{itemize}
		\end{cvsubsection}

		\begin{cvsubsection}{Proyectos personales}{}{}
			\begin{itemize}
				\item \textbf{Prototipo Ray Tracer} (2013). Proyecto personal para implementar técnicas de Ray Tracing para gráficos 3D en tiempo real. C++11, OpenCL y OpenCL-C.
				\item \textbf{Prototipo IDE} (2017). Proyecto personal para el desarrollo de un IDE, cross-platform, como alternativa Open Source a Visual Studio. C++11 con Qt 5 y Win32 API.
				\item \textbf{Prototipo Motor Multimedia} (2017). Proyecto personal para desarrollar un middleware modular y extensible para el desarrollo de juegos y otras aplicaciones multimedia. C++11, OpenGL 3+, y Vulkan.
			\end{itemize}
		\end{cvsubsection}
	\end{cvsection}
	
	\begin{cvsection}{Lenguajes, Tecnologías e Idiomas}
		\begin{cvsubsection}{}{}{}	
			\begin{itemize}
				\item C++; C; MFC; Win32 API; Boost C++ libraries; Java; Swift; Objective-C; Csharp.NET; Windows Forms; WPF; Transact-SQL; JavaScript; Node.JS, TypeScript, Angular 8+; XML; JSON; OpenGL; GLSL; OpenCL; OpenCL-C; Python; Visual Basic 6; Emscripten; ASM.JS / WebAssembly; C++/CLI.
				\item Visual Studio; Visual Studio Code; Microsoft SQL Server; Eclipse; XCode; IntelliJ; Dbeaver.
				\item Git; Subversion; SSH; Unix Scripting; Windows Command Line.
				\item Docker; Kubernetes; Jenkins; Portainer; Heroku.
				\item Inglés Intermedio.
			\end{itemize}
		\end{cvsubsection}
	\end{cvsection}
\end{document}
